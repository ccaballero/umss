\chapter{Fuerzas disipativas}

\section{Ejemplos}

\subsection{Ejemplo 1}
6.9. Una pequeña masa $m_1$, colgada de un resorte espiral ligero de longitud
$r$, longitud normal $r_0$, y constante elástica $k$, puede oscilar como un
péndulo, sobre el plano inclinado rugoso. Utilizando coordenadas polares,
demostrar que para un desplazamiento virtual general (suponiendo que existe
fricción dinámica):

\begin{equation*}
\begin{split}
    \delta W_{total}=\left[
        -k(r-r_0)+
        mg\sen(\alpha)\cos(\theta)-
        \frac{f\dot{r}}{{(\dot{r}^2+r^2{\theta}^2)}^{\frac{1}{2}}}
    \right]\delta_r-\\
    \left[
        mgr\sen(\alpha)\sen(\theta)+
        \frac{fr^2\dot{\theta}}{{(\dot{r}^2+r^2\dot{\theta}^2)}^{\frac{1}{2}}}
    \right]\delta_{\theta}=F_r\delta_r+F_{\theta}\delta_{\theta}
\end{split}
\end{equation*}

En donde, $f=\mu mg\cos(\alpha)$. ¿Con que condiciones pueden ser discontinuas
las fuerzas de fricción?

\subsection{Ejemplo 2}
6.11. Una partícula se mueve en contacto con un tablero horizontal rugoso.
Suponiendo que el coeficiente de fricción para movimiento en la dirección de $X$
es $\mu_x$, y que en la dirección de $Y$ es $\mu_y$, demostrar que las fuerzas
de fricción generalizadas, correspondientes a las coordenadas polares, son:

\begin{equation*}
    F_r=-f(\mu_x\cos(\theta)+\mu_y\sen(\theta))
\end{equation*}
\begin{equation*}
    F_{\theta}=-fr(\mu_y\cos(\theta)-\mu_x\sen(\theta))
\end{equation*}

En donde $f$ es la fuerza normal entre la partícula y el tablero.

\subsection{Ejemplo 3}
Determinar las ecuaciones de movimiento de un niño que resbala por un plano
inclinado.

\subsection{Ejemplo 4}
6.13. Sobre un tablero plano se ha dibujado un rectángulo de lados $2a$ y $2b$.
En cada una de las esquinas del rectángulo se ha clavado una tachuela de cabeza
redonda, pequeña. Se coloca entonces el tablero sobre las tachuelas, en contacto
con una superficie rugosa, plana. Utilizando como coordenadas $x$, $y$ y
$\theta$ ($x$ y $y$ son las coordenadas del centro del rectángulo y $\theta$ el
ángulo entre el lado $2a$ y el eje $X$) y suponiendo que se presenta fricción en
seco, demostrar que $\delta W_{total}$ viene dado por:

\begin{equation*}
\begin{split}
    -\delta W_{total}=f_1\{
        [\dot{x}+r\dot{\theta}\sen(\theta+\beta)]
        [\delta x+r\delta\theta\sen(\theta+\beta)]+\\
        [\dot{y}-r\dot{\theta}\cos(\theta+\beta)]
        [\delta y-r\delta\theta\cos(\theta+\beta)]
    \}\frac{1}{v_1}+\\
    f_2\{
        [\dot{x}-r\dot{\theta}\sen(\theta-\beta)]
        [\delta x-r\delta\theta\sen(\theta-\beta)]+\\
        [\dot{y}+r\dot{\theta}\cos(\theta-\beta)]
        [\delta y+r\delta\theta\cos(\theta-\beta)]
    \}\frac{1}{v_2}+\\
    f_3\{
        [\dot{x}-r\dot{\theta}\sen(\theta+\beta)]
        [\delta x-r\delta\theta\sen(\theta+\beta)]+\\
        [\dot{y}+r\dot{\theta}\cos(\theta+\beta)]
        [\delta y+r\delta\theta\cos(\theta+\beta)]
    \}\frac{1}{v_3}+\\
    f_4\{
        [\dot{x}+r\dot{\theta}\sen(\theta-\beta)]
        [\delta x+r\delta\theta\sen(\theta-\beta)]+\\
        [\dot{y}-r\dot{\theta}\cos(\theta-\beta)]
        [\delta y-r\delta\theta\cos(\theta-\beta)]
    \}\frac{1}{v_4}
\end{split}
\end{equation*}

En donde:

\begin{equation*}
    r^2=a^2+b^2
\end{equation*}
\begin{equation*}
    \tan(\beta)=b/a
\end{equation*}
\begin{equation*}
    v_1={\{\dot{x}^2+\dot{y}^2+r^2\dot{\theta}^2+2r\dot{\theta}\}}^{\frac{1}{2}}
\end{equation*}

Con expresiones semejantes para $v_2$, $v_3$ y $v_4$. $f_1=\mu$ (fuerza normal
sobre la cabeza de la primera tachuela), etc. Se suponen conocidas $f_1$, $f_2$,
$f_3$ y $f_4$.

Nótese que es posible leer el valor de las fuerzas generalizadas, en la
expresión de $\delta W_{total}$.

\subsection{Ejemplo 5}
6.12. Los bloques $a$ y $b$ unidos rígidamente por medio de una varilla ligera,
de longitud $l$, se deslizan en contacto con los bloques $c$ y $d$. El bloque
$e$ puede moverse, sin fricción, a lo largo de la varilla lisa, mientras que el
bloque $c$ puede moverse sobre el eje $X$ y $d$ permanece fijo. Los coeficientes
de fricción de las superficies de contacto, son $\mu_1$, $\mu_2$ y $\mu_3$, como
se índice. Nótese que el valor de las fuerzas normales depende de la posición de
$m_4$.

Suponiendo que el sistema esta en movimiento de tal manera que $\dot{x_2}$ y
$\dot{x}_1$ son ambas positivas, y que $\dot{x}_2>\dot{x}_1$, demostrar que las
fuerzas generalizadas correspondientes a $x_1$, $q_1$ y $q_2$, son:

\begin{equation*}
    F_{x_1}=-[\mu_1(m_1+m_2+m_4-m_4\frac{q_1}{l})+\mu_3(m_3+m_4\frac{q_1}{l})]g
\end{equation*}
\begin{equation*}
    F_{q_2}=-[\mu_2(m_2+m_4-m_4\frac{q_1}{l})+\mu_3(m_3+m_4\frac{q_1}{l})]g
\end{equation*}
\begin{equation*}
    F_{q_1}=0
\end{equation*}

¿Serán aun validas estas expresiones en el caso en que $\dot{x}_1>\dot{x}_2$?
Evidentemente, las fuerzas de este tipo deber ser manejadas con cuidado.

\subsection{Ejemplo 6}
6.1. Una pequeña esfera esta suspendida por medio de una banda de caucho, dentro
de un liquido viscoso. Suponiendo una fuerza viscosa sencilla que actúa sobre la
esfera y que no hay resistencia sobre la banda, demostrar que las fuerzas
viscosas generalizadas, correspondientes a las coordenadas esféricas, $r$,
$\theta$ y $\phi$, son:

\begin{equation*}
    F_r=-a\dot{r}
\end{equation*}
\begin{equation*}
    F_{\theta}=-ar^2\dot{\theta}
\end{equation*}
\begin{equation*}
    F_{\phi}=-ar^2\sen^2(\theta)\dot{\phi}
\end{equation*}

En donde $a$ es la fuerza viscosa por unidad de velocidad que actúa en la
esfera.

\subsection{Ejemplo 7}
6.5. En la figura, el imán de herradura y el disco de cobre están sostenidos por
tres segmentos de cuerda de piano, de tal manera que se forma un péndulo doble
de torsión. Las constantes de torsión de las cuerdas son $C_1$, $C_2$ y $C_3$,
respectivamente. Existe una resistencia viscosa $a_1$, por unidad de velocidad
entre el disco y los bloques de freno $B_1$ y $B_2$, y una resistencia semejante
$a_2$, entre los polos magnéticos y el disco. Demostrar que las ecuaciones del
movimiento del sistema son (suponiendo que $a_1$ y $a_2$ actúan a distancias
radiales $r_1$ y $r_2$):

\begin{equation*}
    I_1\ddot{\theta}_1+
    C_3\theta_1+
    C_2(\theta_1-\theta_2)+
    2a_1r_1^2\dot{\theta}_1+
    2a_2r_2^2(\dot{\theta}_1-\dot{\theta}_2)=0
\end{equation*}
\begin{equation*}
    I_2\ddot{\theta}_2-
    C_2(\theta_1-\theta_2)+
    C_1\theta_2-
    2a_2r_2^2(\dot{\theta}_1-\dot{\theta}_2)=0
\end{equation*}

En donde $I_1$ e $I_2$ son los momentos de inercia del disco y del imán,
respectivamente, y $\theta_1$ y $\theta_2$ son los desplazamientos angulares
correspondientes.

