\chapter{Ecuaciones de \emph{Lagrange} para una partícula}

\section{Introducción}

Ecuación de \emph{Lagrange}:

\begin{equation*}
    \frac{d}{dt}\left(\frac{\partial T}{\partial \dot{q}_r}\right)-
    \frac{\partial T}{\partial q_r}=F_{q_r}
\end{equation*}

Principio de \emph{d'Alembert}:

\begin{equation*}
    m(\ddot{x}\delta x+\ddot{y}\delta y+\ddot{z}\delta z)=
    F_x\delta x+F_y\delta y+F_z\delta z
\end{equation*}

\section{Deducción de la ecuación de \emph{Lagrange}}

\begin{equation*}
    \begin{cases}
        x=x(q_1,q_2)\\
        y=y(q_1,q_2)\\
        z=z(q_1,q_2)
    \end{cases}
\end{equation*}

\begin{equation*}
    \delta_x=\frac{\partial x}{\partial q_1}\delta q_1+
             \frac{\partial x}{\partial q_2}\delta q_2
\end{equation*}
\begin{equation*}
    \delta_y=\frac{\partial y}{\partial q_1}\delta q_1+
             \frac{\partial y}{\partial q_2}\delta q_2
\end{equation*}
\begin{equation*}
    \delta_z=\frac{\partial z}{\partial q_1}\delta q_1+
             \frac{\partial z}{\partial q_2}\delta q_2
\end{equation*}

\begin{equation*}
\begin{split}
    m\left(
        \ddot{x}\left(
            \frac{\partial x}{\partial q_1}\delta q_1+
            \frac{\partial x}{\partial q_2}\delta q_2
        \right)+
        \ddot{y}\left(
            \frac{\partial y}{\partial q_1}\delta q_1+
            \frac{\partial y}{\partial q_2}\delta q_2
        \right)+
        \ddot{z}\left(
            \frac{\partial z}{\partial q_1}\delta q_1+
            \frac{\partial z}{\partial q_2}\delta q_2
        \right)
    \right)=\\
    F_x\left(
        \frac{\partial x}{\partial q_1}\delta q_1+
        \frac{\partial x}{\partial q_2}\delta q_2
    \right)+
    F_y\left(
        \frac{\partial y}{\partial q_1}\delta q_1+
        \frac{\partial y}{\partial q_2}\delta q_2
    \right)+
    F_z\left(
        \frac{\partial z}{\partial q_1}\delta q_1+
        \frac{\partial z}{\partial q_2}\delta q_2
    \right)
\end{split}
\end{equation*}

\begin{equation*}
\begin{split}
    m\left(
        \ddot{x}\frac{\partial x}{\partial q_1}\delta q_1+
        \ddot{y}\frac{\partial y}{\partial q_1}\delta q_1+
        \ddot{z}\frac{\partial z}{\partial q_1}\delta q_1
    \right)+
    m\left(
        \ddot{x}\frac{\partial x}{\partial q_2}\delta q_2+
        \ddot{y}\frac{\partial y}{\partial q_2}\delta q_2+
        \ddot{z}\frac{\partial z}{\partial q_2}\delta q_2
    \right)=\\
    \left(
        F_x\frac{\partial x}{\partial q_1}\delta q_1+
        F_y\frac{\partial y}{\partial q_1}\delta q_1+
        F_z\frac{\partial z}{\partial q_1}\delta q_1
    \right)+
    \left(
        F_x\frac{\partial x}{\partial q_2}\delta q_2+
        F_y\frac{\partial y}{\partial q_2}\delta q_2+
        F_z\frac{\partial z}{\partial q_2}\delta q_2
    \right)
\end{split}
\end{equation*}

\begin{equation*}
\begin{split}
    m\left(
        \left(
            \ddot{x}\frac{\partial x}{\partial q_1}+
            \ddot{y}\frac{\partial y}{\partial q_1}+
            \ddot{z}\frac{\partial z}{\partial q_1}
        \right)\delta q_1
    \right)+
    m\left(
        \left(
            \ddot{x}\frac{\partial x}{\partial q_2}+
            \ddot{y}\frac{\partial y}{\partial q_2}+
            \ddot{z}\frac{\partial z}{\partial q_2}
        \right)\delta q_2
    \right)=\\
    \left(
        F_x\frac{\partial x}{\partial q_1}+
        F_y\frac{\partial y}{\partial q_1}+
        F_z\frac{\partial z}{\partial q_1}
    \right)\delta q_1+
    \left(
        F_x\frac{\partial x}{\partial q_2}+
        F_y\frac{\partial y}{\partial q_2}+
        F_z\frac{\partial z}{\partial q_2}
    \right)\delta q_2
\end{split}
\end{equation*}

Si $q_2=0$:

\begin{equation}
    m\left(
        \left(
            \ddot{x}\frac{\partial x}{\partial q_1}+
            \ddot{y}\frac{\partial y}{\partial q_1}+
            \ddot{z}\frac{\partial z}{\partial q_1}
        \right)\delta q_1
    \right)=
    \left(
        F_x\frac{\partial x}{\partial q_1}+
        F_y\frac{\partial y}{\partial q_1}+
        F_z\frac{\partial z}{\partial q_1}
    \right)\delta q_1
    \label{general1}
\end{equation}

Sabiendo que por la regla de la cadena:

\begin{equation*}
    \frac{d}{dt}\left(\dot{x}\frac{\partial x}{\partial q_1}\right)=
    \ddot{x}\frac{\partial x}{\partial q_1}+
    \dot{x}\frac{d}{dt}\left(\frac{\partial x}{\partial q_1}\right)
\end{equation*}

\begin{equation}
    \ddot{x}\frac{\partial x}{\partial q_1}=
    \frac{d}{dt}\left(\dot{x}\frac{\partial x}{\partial q_1}\right)-
    \dot{x}\frac{d}{dt}\left(\frac{\partial x}{\partial q_1}\right)
    \label{base}
\end{equation}

Por otro lado, derivando $x=x(q_1,q_2)$ con respecto al tiempo:

\begin{equation*}
    \dot{x}=\frac{\partial x}{\partial q_1}\dot{q_1}+
            \frac{\partial x}{\partial q_2}\dot{q_2}
\end{equation*}

Y derivando con respecto a $\dot{q_1}$:

\begin{equation}
    \frac{\partial \dot{x}}{\partial \dot{q_1}}=\frac{\partial x}{\partial q_1}
    \label{primerTermino}
\end{equation}

Por otro lado:

\begin{equation}
    \frac{d}{dt}\left(\frac{\partial x}{\partial q_1}\right)=
    \frac{\partial}{\partial q_1}\left(\frac{dx}{dt}\right)=
    \frac{\partial \dot{x}}{\partial q_1}
    \label{segundoTermino}
\end{equation}

Combinando~\ref{primerTermino} y~\ref{segundoTermino} en~\ref{base}, se obtiene:

\begin{equation}
    \ddot{x}\frac{\partial x}{\partial q_1}=
    \frac{d}{dt}\left(\dot{x}\frac{\partial \dot{x}}{\partial \dot{q_1}}\right)-
    \dot{x}\frac{\partial \dot{x}}{\partial q_1}
    \label{condensado}
\end{equation}

Por otro lado:

\begin{equation}
    \frac{\partial}{\partial q_1}\left(\frac{1}{2}\dot{x}^2\right)=
    \frac{1}{2}2\dot{x}\frac{\partial\dot{x}}{\partial q_1}=
    \dot{x}\frac{\partial\dot{x}}{\partial q_1}
    \label{cinetica1}
\end{equation}

\begin{equation}
    \frac{\partial}{\partial \dot{q}_1}\left(\frac{1}{2}\dot{x}^2\right)=
    \frac{1}{2}2\dot{x}\frac{\partial\dot{x}}{\partial \dot{q}_1}=
    \dot{x}\frac{\partial\dot{x}}{\partial \dot{q}_1}
    \label{cinetica2}
\end{equation}

Combinando~\ref{cinetica1} y~\ref{cinetica2} en~\ref{condensado}:

\begin{equation}
    \ddot{x}\frac{\partial x}{\partial q_1}=
    \frac{d}{dt}\left(\frac{\partial}{\partial \dot{q}_1}
        \left(\frac{1}{2}\dot{x}^2\right)
    \right)-
    \frac{\partial}{\partial q_1}\left(\frac{1}{2}\dot{x}^2\right)
    \label{condensado2}
\end{equation}

Sustituyendo~\ref{condensado2}, agregando las relaciones para $y$ y $z$ y
reagrupando términos en~\ref{general1}:

\begin{equation*}
\begin{split}
    \left[
        \frac{d}{dt}\left(
            \frac{\partial}{\partial\dot{q}_1}\left(
                \frac{1}{2}m(\dot{x}^2+\dot{y}^2+\dot{z}^2)
            \right)
        \right)-
        \frac{\partial}{\partial q_1}\left(
            \frac{1}{2}m(\dot{x}^2+\dot{y}^2+\dot{z}^2)
        \right)
    \right]\delta q_1=\\
    \left(
        F_x\frac{\partial x}{\partial q_1}+
        F_y\frac{\partial y}{\partial q_1}+
        F_z\frac{\partial z}{\partial q_1}
    \right)\delta q_1
\end{split}
\end{equation*}

Considerando que $\frac{1}{2}m(\dot{x}^2+\dot{y}^2+\dot{z}^2)$ representa la
energía cinética ($T$), finalmente se obtiene:

\begin{equation*}
    \frac{d}{dt}\left(
        \frac{\partial T}{\partial\dot{q}_1}
    \right)-
    \frac{\partial T}{\partial q_1}=
    F_x\frac{\partial x}{\partial q_1}+
    F_y\frac{\partial y}{\partial q_1}+
    F_z\frac{\partial z}{\partial q_1}
    \label{lagrange}
\end{equation*}

\section{Determinación de la aceleración}
La componente $a'$ del vector aceleración $a$, sobre una linea cuyos cosenos
directores son: $l$, $m$, $n$, esta dada por:

\begin{equation*}
    a'=\ddot{x}l+\ddot{y}m+\ddot{z}n
\end{equation*}

Si debe hallarse $a'$ en la dirección de la tangente a una curva en el espacio
en un punto $p$ determinado.

\begin{align*}
    l&=\frac{dx}{ds}; & m&=\frac{dy}{ds}; & n&=\frac{dz}{ds}
\end{align*}
\begin{equation*}
    ds^2=dx^2+dy^2+dz^2
\end{equation*}

Entonces, partiendo de las ecuaciones de la curva en el espacio, puede hallarse
$l$, $m$, $n$, supóngase que se desea determinar una expresión general para la
componente de la aceleración en la dirección de la tangente a la coordenada
$q_1$, teniendo en cuenta que las coordenadas generalizadas $q_1$, $q_2$, $q_3$
se relacionan con las rectangulares por medio de:

\begin{align*}
    x&=x(q_1,q_2,q_3,t); & y&=y(q_1,q_2,q_3,t); & z=z(q_1,q_2,q_3,t)
\end{align*}

La linea coordenada de $q_1$ se determina haciendo que permanezcan constantes
$q_2$, $q_3$, $t$. En forma semejante se obtienen las lineas coordenadas de
$q_2$ y $q_3$. Tomando las derivadas se obtienen:

\begin{align*}
    dx&=\frac{\partial x}{\partial q_1}dq_1; &
    dy&=\frac{\partial y}{\partial q_1}dq_1; &
    dz&=\frac{\partial z}{\partial q_1}dq_1
\end{align*}

Por tanto:

\begin{equation*}
    {ds}^2=\left[
        {\left(\frac{\partial x}{\partial q_1}\right)}^2+
        {\left(\frac{\partial y}{\partial q_1}\right)}^2+
        {\left(\frac{\partial z}{\partial q_1}\right)}^2
    \right] dq^2_1
\end{equation*}

En donde $ds$ es un elemento de longitud medido sobre la curva de $q$

\begin{equation*}
    l_1=\frac{dx}{dq_1{\left[
            {\left(\frac{\partial x}{\partial q_1}\right)}^2+
            {\left(\frac{\partial y}{\partial q_1}\right)}^2+
            {\left(\frac{\partial z}{\partial q_1}\right)}^2
        \right]}^\frac{1}{2}}
       =\frac{1}{h_1}\frac{\partial x}{\partial q_1}
\end{equation*}

En donde el significado de $h_1$ es claro, y donde se ha escrito $dx/dq_1$ como
$\partial x/\partial q_1$ puesto que $q_2$, $q_3$ y $t$ se consideran todavía
constantes. En forma semejante
$m_1=\frac{1}{h_1}\frac{\partial y}{\partial q_1}$ y
$n_1=\frac{1}{h_1}\frac{\partial y}{\partial q_1}$. Igualmente los cosenos
directores de la tangente a la curva coordenada de $q_2$, son:

\begin{align*}
    l_2&=\frac{1}{h_2}\frac{\partial x}{\partial q_2}; &
    m_2&=\frac{1}{h_2}\frac{\partial y}{\partial q_2}; &
    n_2&=\frac{1}{h_2}\frac{\partial z}{\partial q_2};
\end{align*}

Designando $a'$ como $a_{q_1}$ la ecuación se convierte en:

\begin{equation*}
    a'=a_{q_1}=\frac{1}{h_1}\left(
        \ddot{x}\frac{\partial x}{\partial q_1}+
        \ddot{y}\frac{\partial y}{\partial q_1}+
        \ddot{z}\frac{\partial z}{\partial q_1}
    \right)
\end{equation*}

Teniendo en cuenta las etapas que condujeron al miembro izquierdo de la ecuación
de \emph{Lagrange}, evidentemente puede escribirse en la siguiente forma:

\begin{equation*}
    a_{q_r}=\frac{1}{h_{q_r}}\left[
        \frac{d}{dt}\left(\frac{\partial T'}{\partial\dot{q}_r}\right)-
        \frac{\partial T'}{\partial q_r}
    \right]
\end{equation*}
\begin{equation*}
    T'=\frac{1}{2}v^2=\frac{1}{2}(\dot{x}^2+\dot{y}^2+\dot{z}^2)
\end{equation*}

\section{Ejemplos}

\subsection{Ejemplo 1}
Determinar las ecuaciones de movimiento de una partícula que se mueve por acción
de una fuerza $F$. Suponer que el movimiento es en un plano.

\subsection{Ejemplo 2}
Determinar las ecuaciones de movimiento de una partícula de masa $m$ por la
acción de una fuerza, la misma que tiene movimiento lineal, y angular, la fuerza
que hace posible el movimiento es $F$.

\subsection{Ejemplo 3}
3.5. Una cuenta puede moverse restringida a una espiral cónica lisa. Suponiendo
que $\rho=az$ y que $\phi=-bz$, en donde $a$ y $b$ son constantes, demostrar que
la ecuación del movimiento es:

\begin{equation*}
    \ddot{z}(a^2+1+a^2b^2z^2)+a^2b^2z\dot{z}^2=-g
\end{equation*}

\subsection{Ejemplo 4}
3.7. La masa $m$ del péndulo esta suspendida por medio de un hilo inextensible
del punto $p$. Este punto puede moverse libremente sobre una horizontal bajo la
acción de los resortes, los cuales tienen una constante $k$. Despreciando la
masa de los resortes, demostrar que el péndulo oscila con un periodo igual a:

\begin{equation*}
    P=2r\sqrt{\frac{mg+2kr}{2kg}}
\end{equation*}

\subsection{Ejemplo 5}
Determinar la frecuencia de las pequeñas oscilaciones de un péndulo envolvente
en el campo gravitacional. El radio $r$ del cilindro y la longitud libre $l_0$
de la soga en su posición de equilibrio son datos conocidos.

\subsection{Ejemplo 6}
Un segmento cilíndrico se balancea sobre un plano rugoso por la acción del campo
gravitacional. Determinar el periodo de oscilación, los datos son: la masa $m$,
el momento inercial $I_s$ con respecto al centro de gravedad y la excentricidad
del centro de gravedad $e$.

\subsection{Ejemplo 7}
3.1. Suponiendo que el movimiento está restringido al plano $Q_1 Q_2$ y que la
gravedad actúa en la dirección negativa del eje vertical $Y$, establézcanse las
ecuaciones del movimiento de un proyectil en función de las coordenadas $q_1$ y
$q_2$.


\subsection{Ejemplo 8}
Determinar las ecuaciones de aceleración para una partícula en coordenadas
esféricas.

\subsection{Ejemplo 9}
Determinar las ecuaciones de aceleración para una partícula en coordenadas
polares.

\subsection{Ejemplo 10}
3.16. Un disco pesado uniforme, de radio $R$, rueda por un plano inclinado sin
resbalar. Los ejes $X$ y $Y$ están adheridos a la superficie del disco con el
origen en el centro. Una partícula de masa $m$ puede moverse libremente en el
plano del disco bajo la acción de la gravedad y de un sistema de resortes cuyos
detalles no es necesario especificar. Por consideraciones elementales puede
decirse que despreciando  la pequeña masa de la partícula, el centro del disco
se mueve con aceleración lineal $a=\frac{2}{3}g\sen(\alpha)$, en donde $\alpha$
es el ángulo de inclinación del plano. Demostrar que la energía cinética de la
partícula en coordenadas polares $r$ y $\theta$ ($\theta$ medido a partir del
eje $X$ adherido al disco) esta dada por: ($\beta$ se mide entre $X$ y una recta
fija normal al plano inclinado)

\begin{equation*}
    T=\frac{1}{2}m[
        a^2t^2+
        \dot{r}^2+
        r^2{(\dot{\beta}-\dot{\theta})}^2+
        2at\dot{r}\sen(\beta-\theta)+
        2atr(\dot{\beta}-\dot{\theta})\cos(\beta-\theta)
    ]
\end{equation*}

en donde $\dot{\beta}=(a/R)t$.

\subsection{Ejemplo 11}
3.21. Un alambre rígido parabólico que gira de una cierta manera alrededor del
eje $Z_2$ al mismo tiempo que la plataforma a la cual se halla adherido el
sistema $X_2$, $Y_2$, $Z_2$ se mueve con aceleración constante $a$, paralela al
eje $Y_1$. La cuenta de masa $m$ puede deslizarse a lo largo del alambre liso
bajo la fuerza de la gravedad. (a) Demostrar que:

\begin{equation*}
    T=\frac{1}{2}m[
        \dot{r}^2+
        r^2\dot{\phi}^2+
        a^2t^2+
        2\dot{r}at\sen(\phi)+
        2r\dot{\phi}at\cos(\phi)+
        4b^2r^2\dot{r}^2
    ]
\end{equation*}

Suponiendo $\dot{\phi}=\omega=\text{constante}$, plantear la ecuación del
movimiento correspondiente a $r$, y demostrar que:

\begin{equation*}
    F_r=-2mgbr
\end{equation*}

