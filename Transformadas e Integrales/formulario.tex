\documentclass[letter,twoside,8pt]{article}

\usepackage[spanish,es-nodecimaldot]{babel}
\usepackage[utf8]{inputenc}

\usepackage{helvet}
\renewcommand{\familydefault}{\sfdefault}
\usepackage[T1]{fontenc}
\usepackage{textcomp}

\usepackage[labelfont=bf]{caption}

\usepackage{adjustbox}
\usepackage{graphicx}
\usepackage{pstricks}

\usepackage{amsmath}
\usepackage{amssymb}
\usepackage{booktabs}
\usepackage{siunitx}
\usepackage{enumitem}

\usepackage{anysize}
\marginsize{3cm}{2cm}{2cm}{3cm}

\setlength{\parskip}{6pt}
\special{papersize=215.9mm,279.4mm}

\usepackage{float}
\usepackage{fancyhdr}
\usepackage{lastpage}
\pagestyle{fancy}
\fancyhf{}
\fancyhead[LO]{Transformadas e Integrales}
\fancyhead[RO]{}
\fancyfoot[CO,CE]{\thepage}

\usepackage[nottoc,notlof,notlot]{tocbibind}

\usepackage[
    pdfauthor={Carlos Eduardo Caballero Burgoa},%
    pdftitle={Transformadas e Integrales},%
    pdfsubject={Apuntes},%
    colorlinks,%
    citecolor=black,%
    filecolor=black,%
    linkcolor=black,%
    urlcolor=black,
    breaklinks]{hyperref}
\usepackage{breakurl}

\DeclareMathOperator{\sgn}{sgn}

\begin{document}

\textbf{SERIE TRIGONOMÉTRICA DE \emph{FOURIER}:}
\begin{equation*}
    f(t)=\frac{a_0}{2}+\sum_{n=1}^{\infty}[a_n\cos(n{\omega_0}t)+b_n\sen(n{\omega_0}t)]
\end{equation*}
\begin{equation*}
    a_0=\frac{2}{T}\int_{0}^{T}f(t){dt}
\end{equation*}
\begin{equation*}
    a_n=\frac{2}{T}\int_{0}^{T}f(t)\cos(n{\omega_0}t){dt}
\end{equation*}
\begin{equation*}
    b_n=\frac{2}{T}\int_{0}^{T}f(t)\sen(n{\omega_0}t){dt}
\end{equation*}

\textbf{FORMULAS ÚTILES:}
\begin{equation*}
    \sen({\pi}n)=0;\quad{n}\in\mathbb{N}
\end{equation*}
\begin{equation*}
    \cos({\pi}n)={(-1)}^n;\quad{n}\in\mathbb{N}
\end{equation*}
\begin{equation*}
    \sen({2\pi}n)=0;\quad{n}\in\mathbb{N}
\end{equation*}
\begin{equation*}
    \cos({2\pi}n)=1;\quad{n}\in\mathbb{N}
\end{equation*}

\textbf{DERIVADAS ÚTILES:}
\begin{equation*}
    \frac{d}{dt}[\arctan(t)]=\frac{1}{t^2+1}t'
\end{equation*}
\begin{equation*}
    \frac{d}{dt}[\ln(t)]=\frac{1}{t}t'
\end{equation*}

\textbf{INTEGRALES ÚTILES:}
\begin{equation*}
    \int{e}^{at}{dt}=\frac{1}{a}e^{at}
\end{equation*}
\begin{equation*}
    \int{t}{e}^{at}{dt}=\frac{t}{a}{e}^{at}-\frac{1}{a^2}{e}^{at}
\end{equation*}
\begin{equation*}
    \int{t^2}{e}^{at}{dt}=\frac{t^2}{a}{e}^{at}-\frac{2t}{a^2}{e}^{at}+\frac{2}{a^3}e^{at}
\end{equation*}
\begin{equation*}
    \int\sen(at){dt}=-\frac{\cos(at)}{a}
\end{equation*}
\begin{equation*}
    \int{t}\sen(at){dt}=-\frac{t}{a}\cos(at)+\frac{1}{a^2}\sen(at)
\end{equation*}
\begin{equation*}
    \int{t^2}\sen(at){dt}=-\frac{t^2}{a}\cos(at)+\frac{2t}{a^2}\sen(at)+\frac{2}{a^3}\cos(at)
\end{equation*}
\begin{equation*}
    \int\cos(at){dt}=\frac{\sen(at)}{a}
\end{equation*}
\begin{equation*}
    \int{t}\cos(at){dt}=\frac{t}{a}\sen(at)+\frac{1}{a^2}\cos(at)
\end{equation*}
\begin{equation*}
    \int{t^2}\cos(at){dt}=\frac{t^2}{a}\sen(at)+\frac{2t}{a^2}\cos(at)-\frac{2}{a^3}\sen(at)
\end{equation*}
\begin{equation*}
    \int\frac{1}{t^2+a^2}dt=\frac{1}{a}\arctan\left(\frac{t}{a}\right)
\end{equation*}
\begin{equation*}
    \int\frac{1}{t^2-a^2}dt=\frac{1}{2a}\ln\left(\frac{t-a}{t+a}\right)
\end{equation*}
\begin{equation*}
    \int\frac{t}{t^2+a^2}dt=\frac{1}{2}\ln(t^2+a^2)
\end{equation*}
\begin{equation*}
    \int\frac{t}{t^2-a^2}dt=\frac{1}{2}\ln(t^2-a^2)
\end{equation*}
\begin{equation*}
    \int\ln(t)dt=t\ln|t|-t
\end{equation*}
\begin{equation*}
    \int{e}^{at}\sen(bt){dt}=\frac{e^{at}}{a^2+b^2}[a\sen(bt)-b\cos(bt)]
\end{equation*}
\begin{equation*}
    \int{e}^{at}\cos(bt){dt}=\frac{e^{at}}{a^2+b^2}[a\cos(bt)+b\sen(bt)]
\end{equation*}

\textbf{SIMETRÍAS DE ONDA:}
\begin{equation*}
\def\arraystretch{1.4}
\begin{array}{@{}lll@{}lll@{}lll@{}}
\toprule
 & a_0 & a_n & b_n\\
\cmidrule(l){1-4}
\text{PAR} & \frac{4}{T}\int_0^{T/2}f(t){dt}
    & \frac{4}{T}\int_0^{T/2}f(t)\cos(n{\omega_0}t){dt}
    & 0\\
\text{IMPAR} & 0
    & 0
    & \frac{4}{T}\int_0^{T/2}f(t)\sen(n{\omega_0}t){dt}\\
\text{S.M.O.} & 0
    & \begin{cases}
        \text{p:}&a_n=0\\
        \text{i:}&a_n=\frac{4}{T}\int_0^{T/2}f(t)\cos(n{\omega_0}t){dt}\\
    \end{cases}
    & \begin{cases}
        \text{p:}&b_n=0\\
        \text{i:}&b_n=\frac{4}{T}\int_0^{T/2}f(t)\sen(n{\omega_0}t){dt}\\
    \end{cases}\\
\text{S.C.O. PAR} & 0
    & \begin{cases}
        \text{p:}&a_n=0\\
        \text{i:}&a_n=\frac{8}{T}\int_0^{T/4}f(t)\cos(n{\omega_0}t){dt}\\
    \end{cases}
    & 0\\
\text{S.C.O. IMPAR} & 0
    & 0
    & \begin{cases}
        \text{p:}&b_n=0\\
        \text{i:}&b_n=\frac{8}{T}\int_0^{T/4}f(t)\sen(n{\omega_0}t){dt}\\
    \end{cases}\\
\bottomrule
\end{array}
\end{equation*}

\textbf{SERIE COMPLEJA DE \emph{FOURIER}:}
\begin{equation*}
    e^{j\theta}=\cos(\theta)+j\sen(\theta)
\end{equation*}
\begin{equation*}
    e^{-j\theta}=\cos(\theta)-j\sen(\theta)
\end{equation*}
\begin{equation*}
    e^{{\pm}j2{\pi}n}=1
\end{equation*}
\begin{equation*}
    e^{{\pm}j\pi}=-1
\end{equation*}
\begin{equation*}
    e^{{\pm}j{\pi}n}=\cos({\pi}n)
\end{equation*}
\begin{equation*}
    \cos(\theta)=\frac{e^{j\theta}+e^{-j\theta}}{2}
\end{equation*}
\begin{equation*}
    \sen(\theta)=\frac{e^{j\theta}-e^{-j\theta}}{2j}
\end{equation*}
\begin{equation*}
f(t)=\sum_{n=-\infty}^{\infty}{c_n}e^{jn{\omega_0}t}
\end{equation*}
\begin{equation*}
    c_n=\frac{1}{T}\int_0^{T}f(t)e^{-jn{\omega_0}t}{dt}
\end{equation*}
\begin{equation*}
    c_0=\frac{1}{T}\int_0^{T}f(t){dt}
\end{equation*}
\begin{equation*}
    a_n=2\,\mathbb{R}e\{c_n\}
\end{equation*}
\begin{equation*}
    b_n=-2\,\mathbb{I}m\{c_n\}
\end{equation*}

\textbf{FUNCIÓN IMPULSO:}
\begin{equation*}
    \int_{-\infty}^{\infty}\delta(t-t_0){dt}=1
\end{equation*}
\begin{equation*}
    \int_{-\infty}^{\infty}\phi(t)\delta(t-t_0){dt}=\phi(t_0)
\end{equation*}
\begin{equation*}
    \int_{-\infty}^{\infty}\phi(t)\delta^{(n)}(t-t_0){dt}={(-1)}^n\phi^{(n)}(t_0)
\end{equation*}
\begin{equation*}
    \delta(at)=\frac{1}{|a|}\delta(t)
\end{equation*}
\begin{equation*}
    t^{n}\delta(t)=0;{n}\in\mathbb{N}
\end{equation*}
\begin{equation*}
    u^{\prime}(t-t_0)=\delta(t-t_0)
\end{equation*}
\begin{equation*}
    \phi(t)\delta(t)=\phi(0)\delta(t)
\end{equation*}
\begin{equation*}
    \delta(-t)=\delta(t)
\end{equation*}

\textbf{SERIE DE \emph{FOURIER} POR DIFERENCIACIÓN:}
\begin{equation*}
    c_n=c^\prime_n+c^{\prime\prime}_n+\cdots+c^{(k)}_n
\end{equation*}
\begin{equation*}
    \gamma^\prime_n=\frac{1}{T}\int_0^{T}f^{\prime}(t)e^{-jn{\omega_0}t}{dt}
\end{equation*}
\begin{equation*}
    c^\prime_n=\frac{\gamma^\prime_n}{jn\omega_0}
\end{equation*}
\begin{equation*}
    \gamma^{\prime\prime}_n=\frac{1}{T}\int_0^{T}f^{\prime\prime}(t)e^{-jn{\omega_0}t}{dt}
\end{equation*}
\begin{equation*}
    c^{\prime\prime}_n=\frac{\gamma^{\prime\prime}_n}{{(jn\omega_0)}^2}
\end{equation*}
\begin{equation*}
    \gamma^{(n)}_n=\frac{1}{T}\int_0^{T}f^{(k)}(t)e^{-jn{\omega_0}t}{dt}
\end{equation*}
\begin{equation*}
    c^{(k)}_n=\frac{\gamma^{(k)}_n}{{(jn\omega_0)}^k}
\end{equation*}

\textbf{TEOREMA DE LA MULTIPLICACIÓN:}
\begin{equation*}
    \frac{1}{T}\int_0^{T}f_1(t)f_2(t){dt}=\sum_{n={-\infty}}^{\infty}{c_1}(n){c_2}(-n)=\sum_{n={-\infty}}^{\infty}{c_1}(-n){c_2}(n)
\end{equation*}

\textbf{TEOREMA DE \emph{PARSEVAL}:}
\begin{equation*}
    \frac{1}{T}\int_0^{T}f^2(t){dt}=c_0^2+\frac{1}{2}\sum_{n=1}^{\infty}(a_n^2+b_n^2)
\end{equation*}

\textbf{TRANSFORMADA DE \emph{FOURIER}:}
\begin{equation*}
    \mathcal{F}\{f(t)\}=F(\omega)=\int_{-\infty}^{\infty}f(t)e^{-j{\omega}t}{dt}
\end{equation*}
\begin{equation*}
    F(\omega)=R(\omega)+jX(\omega)
\end{equation*}
\begin{equation*}
    |F(\omega)|=\sqrt{R^2(\omega)+X^2(\omega)}
\end{equation*}
\begin{equation*}
    \Theta(\omega)=\arctan\left(\frac{X(\omega)}{R(\omega)}\right)
\end{equation*}

\textbf{PROPIEDADES DE LA TRANSFORMADA DE \emph{FOURIER}:}
\begin{equation*}
\def\arraystretch{1.4}
\begin{array}{@{}cll@{}}
\toprule
 1 & \text{Linealidad}
   & \mathcal{F}\{a_1f_1(t)+a_2f_2(t)\}=a_1F_1(\omega)+a_2F_2(\omega)\\
 2 & \text{Cambio de escala}
   & \mathcal{F}\{f(at)\}=\frac{1}{|a|}F\left(\frac{\omega}{a}\right)\\
 3 & \text{Desplazamiento en }\omega
   & \mathcal{F}\{f(t)e^{jat}\}=F(\omega-a)\\
 4 & \text{Desplazamiento en }t
   & \mathcal{F}\{f(t-a)\}=F(\omega)e^{-ja\omega}\\
 5 & \text{Simetría}
   & \mathcal{F}\{F(t)\}=2{\pi}f(-\omega)\\
 6 & \text{Multiplicación}
   & \mathcal{F}\{t^{n}f(t)\}=j^n\frac{d^{(n)}F(\omega)}{d\omega^n};\quad{n}\in\mathbb{N}\\
 7 & \text{Derivada}
   & \mathcal{F}\{f^{(n)}(t)\}={(j\omega)}^{n}F(\omega);\quad{n}\in\mathbb{N}\\
\bottomrule
\end{array}
\end{equation*}

\textbf{FUNCIÓN SIGNO:}
\begin{equation*}
    \sgn^{\prime}(t)=2\delta(t)
\end{equation*}
\begin{equation*}
    |t|^{\prime}=\sgn(t)
\end{equation*}
\begin{equation*}
    \sgn^2(t)=1
\end{equation*}

\textbf{TABLA DE TRANSFORMADAS DE \emph{FOURIER}:}
\begin{equation*}
\def\arraystretch{1.4}
\begin{array}{@{}cll@{}}
\toprule
 & f(t) & F(\omega)=\mathcal{F}\{f(t)\}\\
\cmidrule(l){1-3}
 1 & u(t+a)-u(t-a)
   & \dfrac{2\sen(a\omega)}{\omega}\\
 2 & \dfrac{\sen(at)}{t}
   & \pi[u(\omega+a)-u(\omega-a)]\\
 3 & e^{-at}u(t)\quad{a}>0
   & \dfrac{1}{a+j\omega}\\
 4 & e^{at}u(-t)\quad{a}>0
   & \dfrac{1}{a-j\omega}\\
 5 & e^{-a|t|}\quad{a}>0
   & \dfrac{2a}{a^2+\omega^2}\\
 6 & \dfrac{1}{t^2+a^2}
   & \dfrac{\pi}{a}e^{-a|\omega|}\\
 7 & \delta(t-a)
   & e^{-ja\omega}\\
 8 & e^{jat}
   & 2\pi\delta(\omega-a)\\
 9 & k
   & 2{\pi}k\delta(\omega)\\
10 & \sen(at)
   & j\omega[\delta(\omega+a)-\delta(\omega-a)]\\
11 & \cos(at)
   & \pi[\delta(\omega+a)+\delta(\omega-a)]\\
12 & t^{n}e^{-at}u(t)
   & \dfrac{n!}{{(a+j\omega)}^{n+1}}\quad{n}\in\mathbb{N}\\
13 & u(t)
   & \dfrac{1}{j\omega}+\pi\delta(\omega)\\
14 & \sgn(t)
   & \dfrac{2}{j\omega}\\
15 & |t|
   & -\dfrac{2}{\omega^2}\\
16 & \dfrac{1}{t}
   & -j\pi\sgn(\omega)\\
17 & \dfrac{1}{t^n}
   & \dfrac{j^n\pi{\omega}^{n-1}\sgn(\omega)}{{(-1)}^n(n-1)!}\\
\bottomrule
\end{array}
\end{equation*}

\textbf{FUNCIONES TRIGONOMÉTRICAS DE ARCO DOBLE:}
\begin{equation*}
    \sen(2x)=2\sen(x)\cos(x)
\end{equation*}
\begin{equation*}
    \cos(2x)=2\cos^2(x)-1
\end{equation*}
\begin{equation*}
    \cos(2x)=1-2\sen^2(x)
\end{equation*}

\textbf{FUNCIONES TRIGONOMÉTRICAS DE ARCO TRIPLE:}
\begin{equation*}
    \sen(3x)=3\sen(x)-4\sen^3(x)
\end{equation*}
\begin{equation*}
    \cos(3x)=4\cos^3(a)-3\cos(x)
\end{equation*}

\textbf{FUNCIONES TRIGONOMÉTRICAS DE LA SUMA DE ARCOS:}
\begin{equation*}
    \sen({a}\pm{b})=\sen(a)\cos(b)\pm\sen(b)\cos(a)
\end{equation*}
\begin{equation*}
    \cos({a}\pm{b})=\cos(a)\cos(b)\mp\sen(b)\cos(a)
\end{equation*}

\textbf{FUNCIONES TRIGONOMÉTRICAS DE SUMA A PRODUCTO:}
\begin{equation*}
    \sen(a)+\sen(b)=2\sen\left(\frac{a+b}{2}\right)\cos\left(\frac{a-b}{2}\right)
\end{equation*}
\begin{equation*}
    \sen(a)-\sen(b)=2\cos\left(\frac{a+b}{2}\right)\sen\left(\frac{a-b}{2}\right)
\end{equation*}
\begin{equation*}
    \cos(a)+\cos(b)=2\cos\left(\frac{a+b}{2}\right)\cos\left(\frac{a-b}{2}\right)
\end{equation*}
\begin{equation*}
    \cos(a)-\cos(b)=-2\sen\left(\frac{a+b}{2}\right)\sen\left(\frac{a-b}{2}\right)
\end{equation*}

\textbf{TRANSFORMADA INVERSA DE \emph{FOURIER}:}
\begin{equation*}
    \mathcal{F}\{f(t)\}=F(\omega)\rightarrow\mathcal{F}^{-1}\{F(\omega)\}=f(t)
\end{equation*}

\textbf{TABLA DE TRANSFORMADAS INVERSAS DE \emph{FOURIER}:}
\begin{equation*}
\def\arraystretch{1.4}
\begin{array}{@{}cll@{}}
\toprule
 & F(\omega) & f(t)=\mathcal{F}^{-1}\{F(\omega)\}\\
\cmidrule(l){1-3}
 1 & \dfrac{1}{a+j\omega}
   & e^{-at}u(t)\quad{a}>0\\
 2 & \dfrac{1}{a-j\omega}
   & e^{at}u(-t)\quad{a}>0\\
 3 & \dfrac{2a}{a^2+\omega^2}
   & e^{-a|t|}\quad{a}>0\\
 4 & \frac{1}{\omega}\sen(a\omega)
   & \frac{1}{2}[u(t+a)-u(t-a)]\\
 5 & k
   & k\delta(t)\\
 6 & \dfrac{1}{\omega}
   & \frac{1}{2}j\sgn(t)\\
\bottomrule
\end{array}
\end{equation*}

\textbf{PROPIEDADES DE LA TRANSFORMADA INVERSA DE \emph{FOURIER}:}
\begin{equation*}
\def\arraystretch{1.4}
\begin{array}{@{}cll@{}}
\toprule
 1 & \text{Linealidad}
   & \mathcal{F}^{-1}\{a_1F_1(\omega)+a_2F_2(\omega)\}=a_1f_1(t)+a_2f_2(t)\\
 2 & \text{Desplazamiento en }\omega
   & \mathcal{F}^{-1}\{F(\omega-a)\}=f(t)e^{jat}\\
 3 & \text{Desplazamiento en }t
   & \mathcal{F}^{-1}\{F(\omega)e^{-ja\omega}\}=f(t-a)\\
\bottomrule
\end{array}
\end{equation*}

\textbf{CONVOLUCIÓN:}
\begin{equation*}
    f(t)=f_1(t)*f_2(t)=\int_{-\infty}^{\infty}f_1(\tau)f_2(t-\tau){d\tau}
\end{equation*}

\textbf{PROPIEDADES DE LA CONVOLUCIÓN:}
\begin{equation*}
\def\arraystretch{1.4}
\begin{array}{@{}cll@{}}
\toprule
 1 & \text{Conmutatividad}
   & f_1(t)*f_2(t)=f_2(t)*f_1(t)\\
 2 & \text{Asociatividad}
   & f_1(t)*[f_2(t)*f_3(t)]=[f_1(t)*f_2(t)]*f_3(t)\\
 3 & \text{Distributividad}
   & f_1(t)*[f_2(t)+f_3(t)]=f_1(t)*f_2(t)+f_1(t)*f_3(t)\\
 4 & \text{Función impulso}
   & f_1(t)*\delta(t-t_0)=f_1(t-t_0)\\
 5 & \text{Función escalón unitario}
   & [f_1(t)u(t)]*[f_2(t)u(t)]=\int_0^{t}f_1(\tau)f_2(t-\tau){d\tau}\\
\bottomrule
\end{array}
\end{equation*}

\textbf{TRANSFORMADA DE \emph{FOURIER} Y CONVOLUCIÓN:}
\begin{equation*}
    \mathcal{F}\{f_1(t)*f_2(t)\}=F_1(\omega)F_2(\omega)
\end{equation*}
\begin{equation*}
    \mathcal{F}^{-1}\{F_1(\omega)F_2(\omega)\}=f_1(t)*f_2(t)
\end{equation*}

\textbf{ECUACIONES DIFERENCIALES ORDINARIAS:}
\begin{equation*}
    \mathcal{F}\{f^{\prime}(t)\}=j{\omega}F(\omega)
\end{equation*}
\begin{equation*}
    \mathcal{F}\{f^{\prime\prime}(t)\}={(j\omega)}^{2}F(\omega)
\end{equation*}
\begin{equation*}
    \mathcal{F}\{f^{(n)}(t)\}={(j\omega)}^{n}F(\omega)
\end{equation*}

\textbf{TRANSFORMADA DE \emph{LAPLACE}:}
\begin{equation*}
    \mathcal{L}\{f(t)\}=F(s)=\int_0^{\infty}f(t)e^{-st}{dt}
\end{equation*}

\textbf{PROPIEDADES DE LA TRANSFORMADA DE \emph{LAPLACE}:}
\begin{equation*}
\def\arraystretch{1.4}
\begin{array}{@{}cll@{}}
\toprule
 1 & \text{Linealidad}
   & \mathcal{L}\{a_1f_1(t)+a_2f_2(t)\}=a_1F_1(s)+a_2F_2(s)\\
 2 & \text{Desplazamiento en }s
   & \mathcal{L}\{f(t)e^{at}\}=F(s-a)\\
 3 & \text{Desplazamiento en }t
   & \mathcal{L}\{f(t-a)u(t-a)\}=F(s)e^{-as}\\
   &
   & \mathcal{L}\{f(t)u(t-a)\}=e^{-as}\mathcal{L}\{f(t+a)\}\\
 4 & \text{Multiplicación}
   & \mathcal{L}\{{t}f(t)\}=-\frac{dF(s)}{ds}\\
   &
   & \mathcal{L}\{t^{n}f(t)\}={(-1)}^n\frac{d^{(n)}F(s)}{ds^n}\\
 5 & \text{División}
   & \mathcal{L}\biggl\{\frac{1}{t}f(t)\biggl\}=\int_s^{\infty}F(s){ds}\\
 6 & \text{Derivadas}
   & \mathcal{L}\{f^{\prime}(t)\}=sF(s)-f(0)\\
   &
   & \mathcal{L}\{f^{\prime\prime}(t)\}=s^2F(s)-f(0)s-f^{\prime}(0)\\
   &
   & \mathcal{L}\{f^{\prime\prime\prime}(t)\}=s^3F(s)-f(0)s^2-f^{\prime}(0)s-f^{\prime\prime}(0)\\
 7 & \text{Integrales}
   & \mathcal{L}\biggl\{\int_0^{t}f(t){dt}\biggl\}=\frac{1}{s}F(s)\\
\bottomrule
\end{array}
\end{equation*}

\textbf{TABLA DE TRANSFORMADAS DE \emph{LAPLACE}:}
\begin{equation*}
\def\arraystretch{1.4}
\begin{array}{@{}cll@{}}
\toprule
 & f(t) & F(s)=\mathcal{L}\{f(t)\}\\
\cmidrule(l){1-3}
 1 & k
   & \dfrac{k}{s}\\
 2 & t^n
   & \dfrac{\Gamma(n+1)}{s^{n+1}}\\
   &
   & \dfrac{n!}{s^{n+1}};\quad{n}\in\mathbb{N}\\
 3 & e^{at}
   & \dfrac{1}{s-a}\\
 4 & \sen(at)
   & \dfrac{a}{s^2+a^2}\\
 5 & \cos(at)
   & \dfrac{s}{s^2+a^2}\\
 6 & \senh(at)
   & \dfrac{a}{s^2-a^2}\\
 7 & \cosh(at)
   & \dfrac{s}{s^2-a^2}\\
 8 & u(t-a)
   & \dfrac{1}{s}e^{-as}\\
 9 & \delta(t-a)
   & e^{-at}\\
10 & \dfrac{1}{t}\sen(at)
   & \arctan\left(\dfrac{a}{s}\right)\\
\bottomrule
\end{array}
\end{equation*}

\textbf{FUNCIÓN \emph{GAMMA}:}
\begin{equation*}
    \Gamma(n)=\int_0^{\infty}x^{n-1}e^{-x}{dx}
\end{equation*}

\textbf{PROPIEDADES DE LA FUNCIÓN \emph{GAMMA}:}
\begin{equation*}
\def\arraystretch{1.4}
\begin{array}{@{}cll@{}}
\toprule
 1 & \text{Propiedad 1}
   & \Gamma(n)=(n-1)\Gamma(n-1)\\
   &
   & \Gamma(n)=(n-1)(n-2)(n-3)\dots(n-r)\Gamma(n-r)\\
 2 & \text{Propiedad 2}
   & \Gamma(n)=\frac{\Gamma(n+1)}{n}\\
 3 & \text{Propiedad 3}
   & \Gamma(n)=(n-1)!\\
   &
   & 0!=1\\
 4 & \text{Propiedad 4}
   & \Gamma\left(\frac{1}{2}\right)=\sqrt{\pi}\\
   &
   & \Gamma\left(-\frac{1}{2}\right)=-2\sqrt{\pi}\\
\bottomrule
\end{array}
\end{equation*}

\textbf{TEOREMA DEL VALOR INICIAL Y FINAL:}
\begin{equation*}
    f(0)=\lim_{t\rightarrow{0}}f(t)=\lim_{t\rightarrow\infty}sF(s)
\end{equation*}
\begin{equation*}
    f(\infty)=\lim_{t\rightarrow\infty}f(t)=\lim_{t\rightarrow{0}}sF(s)
\end{equation*}

\textbf{TRANSFORMADA INVERSA DE \emph{LAPLACE}:}
\begin{equation*}
    \mathcal{L}^{-1}\{F(s)\}=f(t);\quad{t}>0
\end{equation*}

\textbf{TABLA DE TRANSFORMADAS INVERSAS DE \emph{LAPLACE}:}
\begin{equation*}
\def\arraystretch{1.4}
\begin{array}{@{}cll@{}}
\toprule
 & F(s) & f(t)=\mathcal{L}^{-1}\{F(s)\};t>0\\
\cmidrule(l){1-3}
 1 & \dfrac{k}{s}
   & k\\
 2 & \dfrac{1}{s^n}
   & \dfrac{t^{n-1}}{\Gamma(n)}\\
   &
   & \dfrac{t^{n-1}}{(n-1)!};\quad{n}\in\mathbb{N}\\
 3 & \dfrac{1}{s-a}
   & e^{at}\\
 4 & \dfrac{1}{s^2+a^2}
   & \dfrac{1}{a}\sen(at)\\
 5 & \dfrac{s}{s^2+a^2}
   & \cos(at)\\
 6 & \dfrac{1}{s^2-a^2}
   & \dfrac{1}{a}\senh(at)\\
 7 & \dfrac{s}{s^2-a^2}
   & \cosh(at)\\
 8 & \arctan\left(\dfrac{a}{s}\right)
   & \dfrac{1}{t}\sen(at)\\
 9 & k
   & k\delta(t)\\
10 & e^{-as}
   & \delta(t-a)\\
\bottomrule
\end{array}
\end{equation*}

\textbf{PROPIEDADES DE LA TRANSFORMADA INVERSA DE \emph{LAPLACE}:}
\begin{equation*}
\def\arraystretch{1.4}
\begin{array}{@{}cll@{}}
\toprule
 1 & \text{Linealidad}
   & \mathcal{L}^{-1}\{a_1F_1(s)+a_2F_2(s)\}=a_1f_1(t)+a_2f_2(t)\\
 2 & \text{Desplazamiento en }s
   & \mathcal{L}^{-1}\{F(s-a)\}=f(t)e^{at}\\
 3 & \text{Desplazamiento en }t
   & \mathcal{L}^{-1}\{F(s)e^{-as}\}=f(t-a)u(t-a)\\
 4 & \text{División por }s
   & \mathcal{L}^{-1}\biggl\{\frac{F(s)}{s}\biggl\}=\int_0^{t}f(t){dt}\\
   &
   & \mathcal{L}^{-1}\biggl\{\frac{F(s)}{s^n}\biggl\}=\int_0^t\int_0^t\dots\int_0^{t}f(t){dt}\dots{dt}{dt}\\
 5 & \text{Derivada}
   & \mathcal{L}^{-1}\{F^{\prime}(s)\}={-t}f(t)\\
   &
   & \mathcal{L}^{-1}\{F^{(n)}(s)\}={(-1)}^{n}t^{n}f(t)\\
\bottomrule
\end{array}
\end{equation*}

\textbf{DESCOMPOSICIÓN EN FRACCIONES PARCIALES:}
\begin{equation*}
    \frac{P(s)}{(s-a_1)(s-a_2)\dots(s-a_n)}
        =\frac{A_1}{s-a_1}+\frac{A_2}{s-a_2}+\cdots+\frac{A_n}{s-a_n}
\end{equation*}
\begin{equation*}
    \frac{P(s)}{{(s-a)}^m{(s-b)}^n}
        =\frac{A_1}{s-a}+\frac{A_2}{{(s-a)}^2}+\cdots+\frac{A_m}{{(s-a)}^m}
        +\frac{B_1}{(s-b)}+\frac{B_2}{{(s-a)}^2}+\cdots+\frac{B_n}{{(s-b)}^n}
\end{equation*}
\begin{equation*}
    \frac{P(s)}{(s^2+{a_1}s+b_1)(s^2+{a_2}s+b_2)}
        =\frac{{A_1}s+B_1}{s^2+{a_1}s+b_1}
        +\frac{{A_2}s+B_2}{s^2+{a_2}s+b_2}
\end{equation*}

\textbf{CONVOLUCIÓN:}
\begin{equation*}
    f(t)=f_1(t)*f_2(t)=\int_{0}^{t}f_1(\tau)f_2(t-\tau){d\tau}
\end{equation*}

\textbf{TRANSFORMADA DE \emph{LAPLACE} Y CONVOLUCIÓN:}
\begin{equation*}
    \mathcal{L}\{f_1(t)*f_2(t)\}=F_1(s)F_2(s)
\end{equation*}
\begin{equation*}
    \mathcal{L}^{-1}\{F_1(s)F_2(s)\}=f_1(t)*f_2(t)
\end{equation*}

\textbf{APLICACIONES DE LA TRANSFORMADA DE \emph{LAPLACE}:}
\begin{equation*}
    \mathcal{L}\{f^{\prime}(t)\}=sF(s)-f(0)
\end{equation*}
\begin{equation*}
    \mathcal{L}\{f^{\prime\prime}(t)\}=s^2F(s)-f(0)s-f^{\prime}(0)
\end{equation*}
\begin{equation*}
    \mathcal{L}\{f^{\prime\prime\prime}(t)\}=s^3F(s)-f(0)s^2-f^{\prime}(0)s-f^{\prime\prime}(0)
\end{equation*}

\textbf{CIRCUITOS RLC:}
\begin{equation*}
    L\left(\frac{d^{2}q}{dt^2}\right)+R\left(\frac{dq}{dt}\right)+\frac{1}{C}(q)=V(t)
\end{equation*}
\begin{equation*}
    L\left(\frac{di}{dt}\right)+R(i)+\frac{1}{C}\left(\int_0^{t}{i}\,dt\right)+V_C(0)=V(t)
\end{equation*}


\end{document}

