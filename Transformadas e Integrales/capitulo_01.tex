\chapter{Series de \emph{Fourier}}

\section{Funciones periodicas}

\begin{figure}
    \centering
    \input{figura_01.tex}
    \caption{Función periodica}
    \label{figura_01}
\end{figure}

Una funcion periodica es aquella cuya grafica se repite infinitas veces, cada
cierto intervalo (\textbf{Figura \ref{figura_01}}).

El menor intervalo de repetición se llama \emph{periodo} ($T$).

Matematicamente una funcion periodica es aquella que verifica:

\begin{equation}
    f(t) = f(t + nT); n \in \mathbb{Z}
\end{equation}

Donde $T$ es el periodo (la menor constante que verifica la igualdad).

\section{Propiedades de la funciones periodicas}

Si: $f(t) = f(t+nT)$.

\subsection{Propiedad 1}
\begin{equation}
    \int_a^b f(t)\,dt = \int_{a+nT}^{b+nT} f(t)\,dt \quad n \in \mathbb{Z}
\end{equation}

\underline{Prueba}:
\begin{equation*}
    \int_a^b f(t)\,dt = \int_{a}^{b} f(t+nT)\,dt
\end{equation*}
\begin{equation*}
    \tau = t + nT
\end{equation*}
\begin{equation*}
    d\tau = dt
\end{equation*}
\begin{equation*}
    \int_a^b f(t)\,dt = \int_{a+nT}^{b+nT} f(\tau)\,d\tau
\end{equation*}

Puede verse graficamente en la \textbf{Figura \ref{figura_02}}.

\begin{figure}
    \centering
    \input{figura_02.tex}
    \caption{Demostración grafica}
    \label{figura_02}
\end{figure}

\subsection{Propiedad 2}

\begin{equation}
    \int_0^b f(t)\,dt = \int_{a+nT}^{b+nT} f(t)\,dt \quad n \in \mathbb{Z}
\end{equation}

\underline{Prueba}:
\begin{equation*}
    \int_a^b f(t)\,dt = \int_{a}^{b} f(t+nT)\,dt
\end{equation*}
\begin{equation*}
    \tau = t + nT
\end{equation*}
\begin{equation*}
    d\tau = dt
\end{equation*}
\begin{equation*}
    \int_a^b f(t)\,dt = \int_{a+nT}^{b+nT} f(\tau)\,d\tau
\end{equation*}

