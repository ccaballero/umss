\documentclass[letter,11pt]{article}

\usepackage[spanish,es-nodecimaldot]{babel}
\usepackage[utf8]{inputenc}

\usepackage{lmodern}
\usepackage[T1]{fontenc}
\usepackage{textcomp}

\usepackage{framed}
\usepackage[svgnames]{xcolor}
\colorlet{shadecolor}{Gainsboro!50}

\usepackage{graphicx}
\usepackage{pstricks}

\usepackage{anysize}
\marginsize{3cm}{2cm}{2cm}{3cm}

\usepackage{siunitx}
\usepackage{amsmath}
\usepackage{array}

\usepackage{fancyhdr}
\usepackage{lastpage}
\pagestyle{fancy}
\fancyhf{}
\fancyhead[LE,RO]{Física Básica III}
\fancyfoot[CO,CE]{\thepage\ de \pageref{LastPage}}

\special{papersize=215.9mm,279.4mm}

\usepackage[
    pdfauthor={Carlos Eduardo Caballero Burgoa},%
    pdftitle={Física Básica III},%
    pdfsubject={2do Parcial},%
    colorlinks,%
    citecolor=black,%
    filecolor=black,%
    linkcolor=black,%
    urlcolor=black,
    breaklinks]{hyperref}
\usepackage{breakurl}

\newcommand{\blankpage}{
\newpage
\thispagestyle{empty}
\mbox{}
\newpage
}

\renewcommand{\arraystretch}{1.2}

\begin{document}

\begin{center}
    {\Large \bf{Segundo parcial}}
\end{center}

\noindent\fbox{%
    \parbox{\textwidth}{%
        Estudiante: CABALLERO BURGOA, Carlos Eduardo \\
        Carrera: Ingeniería Electromecánica \\
        Correo: cijkb.j@gmail.com
    }%
}

\vspace{0.5cm}

\begin{enumerate}
\item La diferencia de potencial en las terminales de una batería es de
$8.4 [V]$ cuando en esta hay una corriente de $1.5 [A]$ de la terminal negativa
a la positiva. Cuando la corriente es de $3.5 [A]$ en la dirección inversa, la
diferencia de potencial es de $10.20 [V]$. ¿Cuál es la fem de la batería?

\begin{itemize}
    \item $ 7.45 [V]$.
    \item $ 8.94 [V]$.
    \item $ 9.26 [V]$.
    \item $10.02 [V]$.
\end{itemize}

Solución: \\

\item Dos focos de $120 [V]$, una de $25 [W]$ y otra de $200 [W]$, se conectaron
en serie a través de una línea de $240 [V]$, pero se quemó de inmediato ...

\begin{itemize}
    \item El de $25 [W]$.
    \item El de $200 [W]$.
    \item Se quemaron los dos.
    \item Ninguno se quemó.
\end{itemize}

Solución: \\

\item Calcule la corriente que fluye por el amperímetro $A$ (de resistencia
interna igual a cero).

\begin{figure}[!h]
\centering
\includegraphics[scale=2.00]{resources/q3.eps}
\end{figure}

\begin{itemize}
    \item $0.0  [A]$.
    \item $0.86 [A]$.
    \item $1.71 [A]$.
    \item $2.06 [A]$.
\end{itemize}

Solución: \\

\item Cierta batería de automóvil puede proporcionar una carga total de
$125 [Ah]$ (amperio-horas) antes de agotarse. Suponiendo que la diferencia de
potencial entre las terminales permanece constante, ¿Cuánto tiempo puede
suministrar energía con una potencia de $110 [W]$?

\begin{itemize}
    \item $13.64 [h]$.
    \item $14.08 [h]$.
    \item $15.39 [h]$.
    \item $16.47 [h]$.
\end{itemize}

Solución: \\

\item La función de la fuerza electromotriz en un circuito consiste en:

\begin{itemize}
    \item Suministrar electrones al circuito.
    \item Elevar el potencial de los electrones.
    \item Disminuir el potencial de los electrones.
    \item Aumentar la rapidez de los electrones.
\end{itemize}

Solución: \\

\item Un calentador (estufa) que opera con una línea de $120 [V]$, tiene una
resistencia de $14 [\Omega]$. ¿Cuánto cuesta hacer funcionar durante
$6 [h] 25 [min]$, si se paga $5.22 [Bs]$ el kWh (kilowat-hora)?

\begin{itemize}
    \item $30.08 [Bs]$.
    \item $32.19 [Bs]$.
    \item $34.45 [Bs]$.
    \item $36.27 [Bs]$.
\end{itemize}

Solución: \\

\item Calcule el potencial del punto $a$ con respecto al punto $b$. Todas las
resistencias están en ohms y todas las fems en volts.

\begin{figure}[!h]
\centering
\includegraphics[scale=1.80]{resources/q7.eps}
\end{figure}

\begin{itemize}
    \item $ 0.11 [V]$.
    \item $ 0.22 [V]$.
    \item $-0.67 [V]$.
    \item $10.22 [V]$.
\end{itemize}

Solución: \\

\item Calcule $V_b - V_a$, la diferencia de potencial de $b$ respecto de $a$.

\begin{figure}[!h]
\centering
\includegraphics[scale=2.00]{resources/q8.eps}
\end{figure}

\begin{itemize}
    \item $ 4.20 [V]$.
    \item $-3.60 [V]$.
    \item $ 2.35 [V]$.
    \item $-1.18 [V]$.
\end{itemize}

Solución: \\

\item Un capacitor de $2 [\mu F]$ inicialmente descargado se conecta en serie
con un resistor de $6000 [\Omega]$ y una fuente de FEM de $90 [V]$. El circuito
se cierra en $t = 0 [s]$. ¿En qué instante la tasa a la que la energía eléctrica
(potencia) se disipa en el resistor es igual a la tasa a la cual la energía
eléctrica se almacena en el capacitor?

\begin{itemize}
    \item $4.58 [ms]$.
    \item $5.97 [ms]$.
    \item $7.23 [ms]$.
    \item $8.32 [ms]$.
\end{itemize}

Solución: \\

\end{enumerate}

\end{document}

